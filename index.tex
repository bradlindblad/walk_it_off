% Options for packages loaded elsewhere
\PassOptionsToPackage{unicode}{hyperref}
\PassOptionsToPackage{hyphens}{url}
\PassOptionsToPackage{dvipsnames,svgnames,x11names}{xcolor}
%
\documentclass[
  letterpaper,
  DIV=11,
  numbers=noendperiod]{scrreprt}

\usepackage{amsmath,amssymb}
\usepackage{iftex}
\ifPDFTeX
  \usepackage[T1]{fontenc}
  \usepackage[utf8]{inputenc}
  \usepackage{textcomp} % provide euro and other symbols
\else % if luatex or xetex
  \usepackage{unicode-math}
  \defaultfontfeatures{Scale=MatchLowercase}
  \defaultfontfeatures[\rmfamily]{Ligatures=TeX,Scale=1}
\fi
\usepackage{lmodern}
\ifPDFTeX\else  
    % xetex/luatex font selection
\fi
% Use upquote if available, for straight quotes in verbatim environments
\IfFileExists{upquote.sty}{\usepackage{upquote}}{}
\IfFileExists{microtype.sty}{% use microtype if available
  \usepackage[]{microtype}
  \UseMicrotypeSet[protrusion]{basicmath} % disable protrusion for tt fonts
}{}
\makeatletter
\@ifundefined{KOMAClassName}{% if non-KOMA class
  \IfFileExists{parskip.sty}{%
    \usepackage{parskip}
  }{% else
    \setlength{\parindent}{0pt}
    \setlength{\parskip}{6pt plus 2pt minus 1pt}}
}{% if KOMA class
  \KOMAoptions{parskip=half}}
\makeatother
\usepackage{xcolor}
\setlength{\emergencystretch}{3em} % prevent overfull lines
\setcounter{secnumdepth}{5}
% Make \paragraph and \subparagraph free-standing
\makeatletter
\ifx\paragraph\undefined\else
  \let\oldparagraph\paragraph
  \renewcommand{\paragraph}{
    \@ifstar
      \xxxParagraphStar
      \xxxParagraphNoStar
  }
  \newcommand{\xxxParagraphStar}[1]{\oldparagraph*{#1}\mbox{}}
  \newcommand{\xxxParagraphNoStar}[1]{\oldparagraph{#1}\mbox{}}
\fi
\ifx\subparagraph\undefined\else
  \let\oldsubparagraph\subparagraph
  \renewcommand{\subparagraph}{
    \@ifstar
      \xxxSubParagraphStar
      \xxxSubParagraphNoStar
  }
  \newcommand{\xxxSubParagraphStar}[1]{\oldsubparagraph*{#1}\mbox{}}
  \newcommand{\xxxSubParagraphNoStar}[1]{\oldsubparagraph{#1}\mbox{}}
\fi
\makeatother


\providecommand{\tightlist}{%
  \setlength{\itemsep}{0pt}\setlength{\parskip}{0pt}}\usepackage{longtable,booktabs,array}
\usepackage{calc} % for calculating minipage widths
% Correct order of tables after \paragraph or \subparagraph
\usepackage{etoolbox}
\makeatletter
\patchcmd\longtable{\par}{\if@noskipsec\mbox{}\fi\par}{}{}
\makeatother
% Allow footnotes in longtable head/foot
\IfFileExists{footnotehyper.sty}{\usepackage{footnotehyper}}{\usepackage{footnote}}
\makesavenoteenv{longtable}
\usepackage{graphicx}
\makeatletter
\def\maxwidth{\ifdim\Gin@nat@width>\linewidth\linewidth\else\Gin@nat@width\fi}
\def\maxheight{\ifdim\Gin@nat@height>\textheight\textheight\else\Gin@nat@height\fi}
\makeatother
% Scale images if necessary, so that they will not overflow the page
% margins by default, and it is still possible to overwrite the defaults
% using explicit options in \includegraphics[width, height, ...]{}
\setkeys{Gin}{width=\maxwidth,height=\maxheight,keepaspectratio}
% Set default figure placement to htbp
\makeatletter
\def\fps@figure{htbp}
\makeatother

\usepackage{booktabs}
\usepackage{caption}
\usepackage{longtable}
\usepackage{colortbl}
\usepackage{array}
\usepackage{anyfontsize}
\usepackage{multirow}
\KOMAoption{captions}{tableheading}
\makeatletter
\@ifpackageloaded{bookmark}{}{\usepackage{bookmark}}
\makeatother
\makeatletter
\@ifpackageloaded{caption}{}{\usepackage{caption}}
\AtBeginDocument{%
\ifdefined\contentsname
  \renewcommand*\contentsname{Table of contents}
\else
  \newcommand\contentsname{Table of contents}
\fi
\ifdefined\listfigurename
  \renewcommand*\listfigurename{List of Figures}
\else
  \newcommand\listfigurename{List of Figures}
\fi
\ifdefined\listtablename
  \renewcommand*\listtablename{List of Tables}
\else
  \newcommand\listtablename{List of Tables}
\fi
\ifdefined\figurename
  \renewcommand*\figurename{Figure}
\else
  \newcommand\figurename{Figure}
\fi
\ifdefined\tablename
  \renewcommand*\tablename{Table}
\else
  \newcommand\tablename{Table}
\fi
}
\@ifpackageloaded{float}{}{\usepackage{float}}
\floatstyle{ruled}
\@ifundefined{c@chapter}{\newfloat{codelisting}{h}{lop}}{\newfloat{codelisting}{h}{lop}[chapter]}
\floatname{codelisting}{Listing}
\newcommand*\listoflistings{\listof{codelisting}{List of Listings}}
\makeatother
\makeatletter
\makeatother
\makeatletter
\@ifpackageloaded{caption}{}{\usepackage{caption}}
\@ifpackageloaded{subcaption}{}{\usepackage{subcaption}}
\makeatother
\ifLuaTeX
  \usepackage{selnolig}  % disable illegal ligatures
\fi
\usepackage{bookmark}

\IfFileExists{xurl.sty}{\usepackage{xurl}}{} % add URL line breaks if available
\urlstyle{same} % disable monospaced font for URLs
\hypersetup{
  pdftitle={Walk It Off},
  pdfauthor={Brad Lindblad},
  colorlinks=true,
  linkcolor={blue},
  filecolor={Maroon},
  citecolor={Blue},
  urlcolor={Blue},
  pdfcreator={LaTeX via pandoc}}

\title{Walk It Off}
\author{Brad Lindblad}
\date{2025-04-30}

\begin{document}
\maketitle

\renewcommand*\contentsname{Table of contents}
{
\hypersetup{linkcolor=}
\setcounter{tocdepth}{2}
\tableofcontents
}
\bookmarksetup{startatroot}

\chapter*{Preface}\label{preface}
\addcontentsline{toc}{chapter}{Preface}

\markboth{Preface}{Preface}

\bookmarksetup{startatroot}

\chapter{Backstory}\label{backstory}

\part{Routine}

When you have something like cancer, you realize that whatever you were
doing \emph{when} you became sick, may have been the reason \emph{why}
you became sick. Wherever you were at physically, mentally, socially,
emotionally -- you need to get away from that state!

Routine is the habits that set our physical, mental, social and
emotional state.

I realized early on that in order to make all the changes that I needed
to make, these things had to fit into a routine that I could check off
each day.

Below I outline the routine and further expand on the components in
later sections.

\subsection*{Weekly tasks}\label{weekly-tasks}
\addcontentsline{toc}{subsection}{Weekly tasks}

\textbf{Exercise \& thermal exposure}

Expand for detail

asdfadf

\begin{itemize}
\item
  6 EWOT sessions on an elliptical

  \begin{itemize}
  \item
    4 days contrast (high and low oxygen)
  \item
    2 days Norweigan 4x4 (VO2 max training)
  \end{itemize}
\item
  3 days strength

  \begin{itemize}
  \tightlist
  \item
    Sand 2.0 program sandbag workouts
  \end{itemize}
\item
  2 cold plunge sessions, \textasciitilde{} 50F
\item
  2 Hockatt or steam room sessions
\end{itemize}

\subsection*{Daily tasks}\label{daily-tasks}
\addcontentsline{toc}{subsection}{Daily tasks}

\textbf{Lymph}

Expand for detail

asdfadf

\begin{itemize}
\item
  Rebounding or shake plate with red light therapy (10 mins)
\item
  Contrast showers
\end{itemize}

\textbf{Breathwork}

Expand for detail

asdfadf

\begin{itemize}
\item
  1-3 sessions of breath work withe the Oxygen Advantage app
\item
  Wim Hof breathing
\end{itemize}

\textbf{Prayer}

\begin{itemize}
\tightlist
\item
  Without ceasing
\end{itemize}

\textbf{Drugs}

Expand for detail

asdfadf

\begin{itemize}
\item
  Tazverik, 800mg 2/x day
\item
  Ivermectin, \textasciitilde80mg (1mg/kg body weight) but visual
  effects force me to cycle this
\end{itemize}

\textbf{Supplements}

Expand for detail

asdfadf

\begin{itemize}
\item
  Causenta-prescribed

  \begin{itemize}
  \item
    DAO Degrade
  \item
    \href{https://causenta.com/store/Gut-Defend-Probiotic-p109023222}{Gut
    Defend}

    \begin{itemize}
    \tightlist
    \item
      Powerful probiotic
    \end{itemize}
  \item
    \href{https://causenta.com/store/HistFx-Vitamin-C-Quercetin-and-Stinging-Nettles-Leaf-p109023244}{Hist
    Fx}

    \begin{itemize}
    \tightlist
    \item
      Regulates histamine response (allergies, etc)
    \end{itemize}
  \item
    Krill Oil (DFH)

    \begin{itemize}
    \item
      To fix an imbalance in my omega-3 to omega-6 ratio
    \item
      Also increase levels of plasmologens (anti-cancer)
    \end{itemize}
  \item
    Menadione (K3)
  \item
    \href{https://aimstore.net/products/optimumbutyrate}{Optimum
    Butyrate}

    \begin{itemize}
    \tightlist
    \item
      Supports gut health and immune function
    \end{itemize}
  \item
    ProResolve Plus
  \item
    \href{https://causenta.com/store/Spore-Defend-Probiotic-p319826021}{Spore
    Defend}

    \begin{itemize}
    \tightlist
    \item
      Broad-spectrum probiotic
    \end{itemize}
  \item
    Vitamin C
  \item
    Vitamin K1

    \begin{itemize}
    \tightlist
    \item
      Testing found that I was very low here
    \end{itemize}
  \item
    Hoxsey-like Formula

    \begin{itemize}
    \tightlist
    \item
      Support detox and lymphatic function
    \end{itemize}
  \item
    \href{https://causenta.com/store/Optimum-Binder-p541360795}{Optimum
    Binder}

    \begin{itemize}
    \tightlist
    \item
      Binds with toxins that are released through detox activities like
      EWOT and hyperthermia
    \end{itemize}
  \item
    \href{https://causenta.com/store/Boswellia-Plus-Salicin-and-Boswellia-p109023254}{Boswellia
    Plus}

    \begin{itemize}
    \tightlist
    \item
      Prescribed for pain, also has anti-cancer effects
    \end{itemize}
  \end{itemize}
\item
  My own research

  \begin{itemize}
  \item
    Vitamin D3 with K2
  \item
    Nanolite xxxx
  \item
    Curcumin/turmeric
  \end{itemize}
\end{itemize}

\chapter{1 Exercise}\label{exercise}

\section{Current protocol}\label{current-protocol}

\subsection{EWOT}\label{ewot}

EWOT or Exercise With Oxygen Therapy is a powerful modality. Here is
what I'm doing.

There are two separate workouts that are done on alternating days, six
days a week.

\begin{table}
\caption*{
{\large Contrast EWOT Workout} \\ 
{\small Performed 4x/week}
} 
\fontsize{12.0pt}{14.4pt}\selectfont
\begin{tabular*}{1\linewidth}{@{\extracolsep{\fill}}rlll}
\toprule
 &  & \multicolumn{2}{c}{3 Sets*} \\ 
\cmidrule(lr){3-4}
Week & Warmup & Low 02 & High O2 \\ 
\midrule\addlinespace[2.5pt]
1 & 5min & 1min & 4min \\ 
2 & 5min & 2min & 3min \\ 
3 & 5min & 3min & 2min \\ 
4 & 5min & 4min & 1min \\ 
\bottomrule
\end{tabular*}
\begin{minipage}{\linewidth}
* Perform the Low O2 then High O2 sections = 1 set. Do that 3x\\
\end{minipage}
\end{table}

\begin{table}
\caption*{
{\large Norweigan 4x4} \\ 
{\small Performed 2x/week}
} 
\fontsize{12.0pt}{14.4pt}\selectfont
\begin{tabular*}{1\linewidth}{@{\extracolsep{\fill}}llll}
\toprule
{\bfseries \cellcolor[HTML]{D9D9D9}{Activity}} & {\bfseries \cellcolor[HTML]{D9D9D9}{Instructions}} & {\bfseries \cellcolor[HTML]{D9D9D9}{Duration}} & {\bfseries \cellcolor[HTML]{D9D9D9}{Intensity}} \\ 
\midrule\addlinespace[2.5pt]
\multicolumn{4}{>{\raggedright\arraybackslash}m{1\linewidth}}{{\bfseries \cellcolor[HTML]{B3B3B3}{1. WARMUP}}} \\[2.5pt] 
\midrule\addlinespace[2.5pt]
{\bfseries \cellcolor[HTML]{E6F2FF}{Easy warmup}} & {\cellcolor[HTML]{E6F2FF}{Moderate activity with oxygen}} & {\cellcolor[HTML]{E6F2FF}{10 minutes}} & {\cellcolor[HTML]{E6F2FF}{Low (40-50\%)}} \\ 
\midrule\addlinespace[2.5pt]
\multicolumn{4}{>{\raggedright\arraybackslash}m{1\linewidth}}{{\bfseries \cellcolor[HTML]{B3B3B3}{2. SPRINT CLUSTERS (Repeat 3-5 clusters)}}} \\[2.5pt] 
\midrule\addlinespace[2.5pt]
{\bfseries \cellcolor[HTML]{FFECEC}{Sprint Interval}} & {\cellcolor[HTML]{FFECEC}{80-95\% effort with oxygen}} & {\cellcolor[HTML]{FFECEC}{4 minutes}} & {\bfseries \cellcolor[HTML]{FFECEC}{\textcolor[HTML]{CC0000}{High (80-95\% max)}}} \\ 
{\bfseries \cellcolor[HTML]{FFF2CC}{Rest Interval}} & {\cellcolor[HTML]{FFF2CC}{Light recovery effort with oxygen}} & {\cellcolor[HTML]{FFF2CC}{3 minutes}} & {\cellcolor[HTML]{FFF2CC}{Low (40-50\%)}} \\ 
{\bfseries \cellcolor[HTML]{F0F0F0}{Repeat}} & {\cellcolor[HTML]{F0F0F0}{Perform 4 sets}} & {\cellcolor[HTML]{F0F0F0}{-}} & {\cellcolor[HTML]{F0F0F0}{-}} \\ 
\bottomrule
\end{tabular*}
\end{table}

\section{Past protocols}\label{past-protocols}

March - May 2025

\subsection{EWOT}\label{ewot-1}

EWOT or Exercise With Oxygen Therapy is a powerful modality. Here is
what I'm doing.

There are two separate workouts that are done on alternating days, seven
days a week.

\subsubsection{Workout A:}\label{workout-a}

\begin{table}
\caption*{
{\large Tabata EWOT Workout A} \\ 
{\small High-Intensity Interval Training with Oxygen Protocol}
} 
\fontsize{12.0pt}{14.4pt}\selectfont
\begin{tabular*}{1\linewidth}{@{\extracolsep{\fill}}llll}
\toprule
{\bfseries \CELLCOLOR[HTML]{D9D9D9}{\TEXTCOLOR[HTML]{A9A9A9}{ACTIVITY}}} & {\bfseries \CELLCOLOR[HTML]{D9D9D9}{\TEXTCOLOR[HTML]{A9A9A9}{INSTRUCTIONS}}} & {\bfseries \CELLCOLOR[HTML]{D9D9D9}{\TEXTCOLOR[HTML]{A9A9A9}{DURATION}}} & {\bfseries \CELLCOLOR[HTML]{D9D9D9}{\TEXTCOLOR[HTML]{A9A9A9}{INTENSITY}}} \\ 
\midrule\addlinespace[2.5pt]
\multicolumn{4}{>{\raggedright\arraybackslash}m{1\linewidth}}{{\bfseries \cellcolor[HTML]{B3B3B3}{1. WARMUP}}} \\[2.5pt] 
\midrule\addlinespace[2.5pt]
{\cellcolor[HTML]{E6F2FF}{Without Oxygen}} & {\cellcolor[HTML]{E6F2FF}{Moderate activity}} & {\cellcolor[HTML]{E6F2FF}{2 minutes}} & {\cellcolor[HTML]{E6F2FF}{Low (40-50\%)}} \\ 
{\cellcolor[HTML]{E6F2FF}{With Oxygen}} & {\cellcolor[HTML]{E6F2FF}{Moderate activity with oxygen}} & {\cellcolor[HTML]{E6F2FF}{2 minutes}} & {\cellcolor[HTML]{E6F2FF}{Low (40-50\%)}} \\ 
\midrule\addlinespace[2.5pt]
\multicolumn{4}{>{\raggedright\arraybackslash}m{1\linewidth}}{{\bfseries \cellcolor[HTML]{B3B3B3}{2. SPRINT CLUSTERS (Repeat 3-5 clusters)}}} \\[2.5pt] 
\midrule\addlinespace[2.5pt]
{\cellcolor[HTML]{FFECEC}{Sprint Interval}} & {\cellcolor[HTML]{FFECEC}{All-out maximum effort}} & {\cellcolor[HTML]{FFECEC}{20 seconds}} & {\cellcolor[HTML]{FFECEC}{\textcolor[HTML]{CC0000}{Maximum (100\%)}}} \\ 
{\cellcolor[HTML]{F0F0F0}{Rest Interval}} & {\cellcolor[HTML]{F0F0F0}{Complete rest}} & {\cellcolor[HTML]{F0F0F0}{10 seconds}} & {\cellcolor[HTML]{F0F0F0}{None (0\%)}} \\ 
{\cellcolor[HTML]{FFF2CC}{Repeat}} & {\cellcolor[HTML]{FFF2CC}{Repeat sprint/rest intervals 3-5 times to complete one cluster}} & {\cellcolor[HTML]{FFF2CC}{\textasciitilde{}2.5 minutes per complete cluster}} & {\cellcolor[HTML]{FFF2CC}{Alternating}} \\ 
{\cellcolor[HTML]{FFF2CC}{Recovery Between Clusters}} & {\cellcolor[HTML]{FFF2CC}{Moderate pace after completing each cluster}} & {\cellcolor[HTML]{FFF2CC}{2 minutes}} & {\cellcolor[HTML]{FFF2CC}{Moderate (40-60\%)}} \\ 
\midrule\addlinespace[2.5pt]
\multicolumn{4}{>{\raggedright\arraybackslash}m{1\linewidth}}{{\bfseries \cellcolor[HTML]{B3B3B3}{3. STEADY STATE}}} \\[2.5pt] 
\midrule\addlinespace[2.5pt]
{\cellcolor[HTML]{E6F2FF}{Calorie Target}} & {\cellcolor[HTML]{E6F2FF}{Continue at normal steady state}} & {\cellcolor[HTML]{E6F2FF}{Until reaching 500 calories}} & {\cellcolor[HTML]{E6F2FF}{Moderate (60-70\%)}} \\ 
\bottomrule
\end{tabular*}
\end{table}

This workout is tough. Use contrast as well (breath low-oxygen during
sprints).

\subsubsection{Workout B:}\label{workout-b}

\begin{table}
\caption*{
{\large Steady State EWOT Workout B} \\ 
{\small Steady State Training with Oxygen Protocol}
} 
\fontsize{12.0pt}{14.4pt}\selectfont
\begin{tabular*}{1\linewidth}{@{\extracolsep{\fill}}llll}
\toprule
{\bfseries \CELLCOLOR[HTML]{D9D9D9}{\TEXTCOLOR[HTML]{A9A9A9}{ACTIVITY}}} & {\bfseries \CELLCOLOR[HTML]{D9D9D9}{\TEXTCOLOR[HTML]{A9A9A9}{INSTRUCTIONS}}} & {\bfseries \CELLCOLOR[HTML]{D9D9D9}{\TEXTCOLOR[HTML]{A9A9A9}{DURATION}}} & {\bfseries \CELLCOLOR[HTML]{D9D9D9}{\TEXTCOLOR[HTML]{A9A9A9}{INTENSITY}}} \\ 
\midrule\addlinespace[2.5pt]
\multicolumn{4}{>{\raggedright\arraybackslash}m{1\linewidth}}{{\bfseries \cellcolor[HTML]{B3B3B3}{1. WARMUP}}} \\[2.5pt] 
\midrule\addlinespace[2.5pt]
{\cellcolor[HTML]{E6F2FF}{With Oxygen}} & {\cellcolor[HTML]{E6F2FF}{Moderate activity with oxygen}} & {\cellcolor[HTML]{E6F2FF}{2 minutes}} & {\cellcolor[HTML]{E6F2FF}{Low (40-50\%)}} \\ 
\midrule\addlinespace[2.5pt]
\multicolumn{4}{>{\raggedright\arraybackslash}m{1\linewidth}}{{\bfseries \cellcolor[HTML]{B3B3B3}{2. STEADY STATE}}} \\[2.5pt] 
\midrule\addlinespace[2.5pt]
{\cellcolor[HTML]{FFF2CC}{With Oxygen}} & {\cellcolor[HTML]{FFF2CC}{Moderate - Heavy activity with oxygen}} & {\cellcolor[HTML]{FFF2CC}{Until reaching 500 - 1,000 calories}} & {\cellcolor[HTML]{FFF2CC}{Moderate (40-60\%)}} \\ 
\bottomrule
\end{tabular*}
\end{table}

Also use contrast here, alternating low oxygen with high.

\chapter{2 Breathing}\label{breathing}

\chapter{3 Prayer}\label{prayer}

\chapter{4 Diet}\label{diet}

\part{Causenta Visits}

\chapter{March 2025}\label{march-2025}

\part{Faith}

\part{Data}

\chapter{1 Cardio}\label{cardio}

\chapter{2 Strength}\label{strength}

\chapter{3 Recovery}\label{recovery}



\end{document}
